%\section{Current Limitations, Conclusions, and Future Work}
\label{sec:limitations}
%\label{sec:conclusions}

%\section{Current Limitations, Conclusions, and Future Work}

To enhance the explainability and usability of AGREE-generated counterexamples, we developed AGREE-Dog, the first open-source conversational copilot specifically integrating neuro-symbolic methods with AGREE's formal verification tools within the OSATE environment. AGREE-Dog produces intuitive, natural-language explanations for complex counterexamples, significantly reducing human effort and cognitive load required for formal model repairs. Our experimental evaluation demonstrates AGREE-Dog's feasibility and effectiveness at realistic MBSE scales—handling scenarios spanning tens of thousands of tokens without notable performance degradation. These initial results provide strong evidence for the practical utility and scalability of neuro-symbolic methods, highlighting significant potential for broader educational and industrial adoption. AGREE-Dog is publicly accessible on GitHub and is scheduled for inclusion in an upcoming OSATE release.

Despite these encouraging outcomes, several avenues for future improvement and exploration remain. We intend to continue evaluating AGREE-Dog on increasingly sophisticated and complex system models and formal specifications. %Additional AGREE functionalities may significantly benefit from generative AI integration as well; promising areas include automatically formalizing AGREE contracts from natural-language requirements and assisting users with model modifications to ensure contract conformance.

Furthermore, ongoing developments in large-context language models (e.g., GPT-4.1’s 1-million-token context window) offer substantial opportunities to explore more autonomous decision-making frameworks, including reinforcement learning-driven judge-router-worker agentic architectures. Such systems could dynamically and autonomously select optimal repair strategies, further reducing manual intervention. Additionally, extending AGREE-Dog’s capabilities to emerging modeling standards, such as SysML v2, represents a key future goal, especially considering that no current SysML v2 tools support comprehensive compositional reasoning or temporal logic analysis comparable to AGREE.

Lastly, the integration of our evaluation workflow into INSPECTA’s DevOps Assurance Dashboard will facilitate continuous monitoring, displaying metrics such as model modifications, counterexample handling efficiency, and AGREE usage statistics. This integration aims to quantify the tangible benefits of more explainable counterexamples, driving targeted improvements in usability and overall user experience.

We look forward to exploring these directions in future work and reporting further advancements toward integrating neuro-symbolic verification approaches in MBSE.


%In this paper, we introduced AGREE-Dog, the first open-source conversational copilot designed explicitly to integrate neuro-symbolic techniques with formal verification for the Architecture Analysis and Design Language (AADL). Seamlessly integrated into the OSATE2 environment, AGREE-Dog facilitates intuitive, traceable, and formally valid explanations and system repairs, significantly reducing cognitive effort and manual intervention. Our evaluation provided promising evidence for AGREE-Dog’s effectiveness within typical Model-Based Systems Engineering (MBSE) scenarios, successfully handling verification tasks involving tens of thousands of tokens without noticeable performance degradation. Given these results, AGREE-Dog demonstrates clear potential not only for practical engineering contexts but also as a tool for education and training. AGREE-Dog is available in beta via GitHub and will appear in an upcoming OSATE2 release.

%However, several open challenges remain. Achieving greater autonomy, particularly through reinforcement learning (RL) techniques to enhance automated strategy recommendation, is an important ongoing research direction. Recent advances, such as GPT-4.1’s extended 1-million-token context window, create new opportunities for exploring RL-driven decision-making and multi-agent solver selection to further automate complex verification workflows. Additionally, migrating AGREE-Dog capabilities toward SysML v2 represents another promising avenue, potentially broadening industrial adoption by enabling compositional reasoning and temporal logic analysis within emerging modeling frameworks.